\chapter{Literature Review}

\section{Relevant Research}

The following section addresses the relevance of my application and the importance of user research in its development. Specifically, it explores why understanding and improving student mood and well-being is crucial for universities. To this end, various scientific studies and publications were reviewed, along with contributions from health-related organizations. These studies aimed to identify the primary causes behind student struggles, gathering critical insights into the factors affecting their emotional state. Additionally, a thorough analysis of similar applications in the market was conducted to evaluate existing features, limitations, and opportunities for innovation. This research sought to pinpoint gaps in current solutions and identify novel ideas that could enhance the user experience. Moreover, a targeted survey was carried out to gather specific user needs and preferences, ensuring that the app's design directly addresses the challenges faced by students.\vspace{5mm} \\
In the course of my user research, I found numerous sources that prove that student well-being is directly related to their productivity. Numerous universities globally invest resources in understanding the intricate connection between students' mental health, stress levels, and academic performance. Additionally, beyond academia, researchers from various fields explore well-being across demographics, employing diverse methodologies such as surveys and longitudinal studies.

\FloatBarrier
\begin{table}[ht]
\centering
\begin{tabular}{|p{4cm}|p{10cm}|}
\hline
\textbf{Research} & \textbf{Summary} \\ \hline
University of Applied Sciences, Northern Netherlands. & This study focused on understanding the key factors affecting student well-being during the COVID-19 pandemic. It identified mental health issues such as anxiety, academic pressure, and inadequate support as major challenges for students. \\ \hline
Renmin University of China, Haidian, Beijing, China. & The research tracked students over four years to evaluate their mental health using the DASS-21 scale. It revealed fluctuations in anxiety, depression, and stress levels, with anxiety peaking in early academic years and improving in later years. \\ \hline
Indonesian Private Universities. & This study examined the relationship between students' emotional states (positive and negative moods) and their academic performance. It found that negative mood significantly hinders learning, while positive mood had no strong effect. The learning process strongly impacted academic outcomes. \\ \hline
The Student Well-being Model. & The study presents a comprehensive model of student well-being, integrating emotional, academic, and social dimensions. It emphasizes the need for a holistic approach to supporting student well-being. \\ \hline
Hascher's Research. & This research highlights the lack of specific tools to measure student well-being and introduces innovative methods such as multi-faceted questionnaires and emotion diaries to assess well-being comprehensively. \\ \hline
\end{tabular}
\caption{Summary of Key Researches on Student Well-being}
\label{tab:research_summary}
\end{table}
\vspace{5mm}
\FloatBarrier

\subsection{University of Applied Sciences, Northern Netherlands}

Researchers aimed to understand the key factors affecting student well-being, particularly during the challenging period of the COVID-19 pandemic \cite{research-1}. Their goal was to identify the main causes behind the mental and emotional struggles students face, including issues such as anxiety, academic pressure, inadequate support from the university, physical and mental health problems, and social difficulties. By focusing on these factors, the study sought to provide insights into how universities can better support student well-being and address the multifaceted challenges that impact their academic and personal lives.

\vspace{5mm}

\noindent \textbf{Method} \\
A total of 113 students signed up for interviews, selected through purposive sampling based on criteria like full-time status, well-being problems, and diverse representation. They were surely informed from researchers (intranet messages) and Heads of School. Participants were from various academies, study years, and genders. Interviews, lasting 28 to 52 minutes, explored student well-being and related factors. Consent was obtained, and interviews were video recorded. The semi-structured guide covered topics such as defining well-being and factors influencing it. Data collection continued until saturation, with 27 students interviewed.

\vspace{5mm}

\noindent \textbf{Analysis} \\
Thematic analysis was used to interpret data, involving verbatim transcription, member checks, and independent coding by three researchers. Coding was initially done line by line using an open coding approach, with the possibility of applying multiple codes to a passage. Codes were then organized into categories based on data and literature concepts, forming themes. Categories, themes, and data allocation were verified in the final phase by two other researchers to ensure clarity and consistency. Discussions resolved any discrepancies until an agreement was reached.

\vspace{5mm}

\noindent \textbf{Results} \\
Participated five male students, twenty-one female students, and one non-binary student, all aged between 17–24 years, participated. There was revealed a diverse range of self-reported well-being issues among the participating students. These included challenges such as inadequate support from the school, anxiety disorders, family-related issues, physical health problems, symptoms of depression, gender dysphoria, fatigue, and performance anxiety. Additionally, factors like stress, planning difficulties, study pressure, loneliness, perfectionism, and getting used to studying were frequently mentioned by the students. Some participants highlighted the impact of life phase-related problems, ADHD, and drug use on their well-being. The findings underscore the complexity and multifaceted nature of the well-being issues faced by students, emphasizing the need for a holistic approach in addressing these concerns, both within and beyond the academic environment.

\vspace{5mm}

\noindent \textbf{Students' opinions} \\
They expressed diverse views on student well-being. Initially, some found it challenging to define, but as interviews progressed, themes emerged. Well-being was associated with managing stress, achieving balance in academic and personal life, and acknowledging the effort-achievement ratio. Stress and resilience were central themes, with students emphasizing the need to cope with challenges. Well-being was seen as a combination of mental, physical, and social aspects, with support from the university playing a crucial role. Students acknowledged the fluctuating nature of well-being and its significant impact on academic performance.

\vspace{5mm}

\noindent \textbf{Main Factors} \\
The main factors influencing student well-being include self-regulation, perfectionism, motivation levels, ability to plan, and study achievements at the individual level. \textbf{Fellow students} contribute to well-being through practical support, idea exchange, enjoyment, and fostering a community atmosphere. \textbf{Tutors} play a crucial role with good relationships, trustworthiness, accessibility, and time availability. \textbf{Attitudes} and \textbf{behaviors} such as empathy, guidance, and personal attention also influence well-being. \textbf{Teachers} impact well-being through good relationships, accessibility, informal contact, community atmosphere, and attitudes like reassurance, understanding, and recognition. Study-related factors include clear communication, flexibility, workload, and the scale of education. The \textbf{university's} support facilities and community atmosphere are significant. \textbf{Peers} outside the university contribute to well-being through enjoyment, idea exchange, understanding, and recognition. \textbf{Family} support, idea exchange, and enjoyment are also key factors.

\subsection{Renmin University of China, Haidian, Beijing, China}

The university conducted research on students' well-being, recognizing the importance of mental health in maintaining overall health \cite{research-2}. The study addresses the global rise in depression and anxiety cases, particularly among college students. College is identified as a critical period for shaping values, and students' emotional well-being is linked to various factors.

\vspace{5mm}

\noindent \textbf{Method} \\
This study employed data from the ``Beijing College Student Panel Survey'' within the ``China Education Panel Survey'', focusing on the 2008 cohort of students tracked for four years from 2009 to 2012. Sampling involved 15 universities, and the effective sample size was 1401 students. The investigation utilized the Depression Anxiety Stress Scales-21 (DASS-21) to assess psychological well-being, a self-report measure known for its reliability. The DASS-21 includes three scales for depression, anxiety, and stress, each comprising seven items. Scores are calculated by summing corresponding item scores. Notably, the DASS-21's short version requires scores to be multiplied by two for comparison with conventional severity ratings. The study reported good validity for the DASS measurement, supported by scale reliability coefficients of 0.813 for depression, 0.766 for anxiety, and 0.812 for stress. Overall, the methodology involved a comprehensive survey approach, online and on-site rounds, and rigorous measures for assessing students' mental health across their college years.

\vspace{5mm}

\noindent \textbf{Results} \\
The results from the study, provide insights into the mental well-being of Chinese college students across four academic years. The average scores for depression and stress, ranging between 7.22 and 7.79 for depression and 9.53 and 11.68 for stress, consistently fell within the normal range based on cutoff values. However, anxiety scores in the first three years slightly surpassed the normal threshold of 7, with mean scores of 7.40, 7.24, and 7.10, indicating above-average anxiety levels. The previously mentioned statistic numbers, as they increase they reflect more and more the severity of depressive and stress-related symptoms. Interestingly, anxiety levels seemed to decrease in the senior year, with a mean score of 6.63. Despite above-normal anxiety levels in the initial years, students maintained mental health with depression and stress scores in the normal range. The study suggests a fluctuation in mental states, with students experiencing higher stress and depression in the sophomore year, while improvements were observed in the last two years. These findings shed light on the dynamic nature of college students' mental well-being over their academic journey, emphasizing the need for targeted interventions and support to enhance mental health throughout their college experience.

\vspace{5mm}

\noindent \textbf{Discussion and Conclusion} \\
In conclusion, the study sheds light on the mental well-being of Chinese college students over four academic years. Freshmen and sophomores exhibited more mental health challenges, possibly stemming from adjustment issues and increased study pressures. Financial concerns were identified as a significant contributor to anxiety, with Chinese students having relatively lower financial burdens compared to their UK counterparts. Notably, the study revealed differences in psychological well-being trends between China and the US, emphasizing the need for tailored interventions. The longitudinal approach strengthened the credibility of the findings, highlighting grade-related disparities. Key conclusions include the average mental health of Chinese college students, their vulnerability to anxiety in the initial years, and the improvement in psychological well-being over time. The study recommends targeted psychological guidance for different academic years, emphasizing the importance of addressing anxiety in freshmen and enhancing well-being support for sophomores. Future research may explore students' mental changes after entering the workforce for further insights into the development of psychological well-being counseling programs in college.

\subsection{Research on Indonesian Private Universities}

The study explores the relationship between students' emotional states — positive and negative moods — and their academic performance in the context of higher education \cite{research-3}. The research aims to uncover how emotional factors such as mood impact the learning process and, ultimately, students' performance in assessments.

\vspace{5mm}

\noindent \textbf{Method} \\
The study investigated how university students' emotional states—both positive and negative moods—impact their learning process and subsequent academic performance. A total of 116 questionnaires were distributed to students at Indonesian private universities, with 106 valid responses collected (86\% response rate). To minimize external influences on their emotional states, the survey was conducted a week before the students' midterm exams.\\
The study used Likert scale-based questionnaires to measure:

\begin{itemize}
    \item \textbf{Positive and Negative Moods}: Each mood was assessed using six items, rated on a scale from 1 (strongly disagree) to 5 (strongly agree).
    \item \textbf{Learning Process}: Self-reported metrics included attendance, assignment submission, pre-class preparation, post-class review, group discussions, and consultations with teachers, also measured on a 5-point scale.
    \item \textbf{Academic Performance}: This was proxied using midterm examination scores.
\end{itemize}

\noindent The study aimed to assess whether students' moods (positive or negative) influenced their engagement with learning activities, which in turn could predict their exam performance. The study used \textbf{structural equation modeling (SEM)} to explore these relationships and analyzed the data for reliability and validity using techniques like convergent validity, composite reliability, and \textbf{average variance extracted (AVE)}.

\vspace{5mm}

\noindent \textbf{Results} \\
The results showed some interesting patterns:

\begin{itemize}
    \item \textbf{Positive Mood}: Contrary to the hypothesis, positive mood did not significantly influence the learning process. Students in a good mood were not necessarily better at engaging with their academic tasks.
    \item \textbf{Negative Mood}: The data revealed a significant negative relationship between a bad mood and the learning process. Students experiencing negative emotions like sadness or stress were less engaged in learning activities, which directly impacted their academic preparation.
    \item \textbf{Learning Process and Academic Performance}: There was a strong positive relationship between the quality of the learning process and academic performance. Students who participated actively in class discussions, submitted assignments, and reviewed study materials performed better in their midterm exams.
\end{itemize} 

\vspace{5mm}

\noindent \textbf{Conclusion} \\
The study's findings highlight the complexity of the relationship between emotions and academic performance:

\begin{itemize}
    \item \textbf{Positive mood}, while often considered beneficial, did not necessarily promote better learning outcomes in this study.
    \item \textbf{Negative mood} had a clear negative impact on learning, suggesting that addressing emotional difficulties could improve academic engagement.
    \item \textbf{Learning behaviors} remained a critical factor in determining academic success, emphasizing the importance of promoting strong study habits among students regardless of their emotional state.
\end{itemize} 

\noindent The study also provided several practical recommendations:

\begin{itemize}
    \item \textbf{For educators}: Teachers and lecturers should focus on creating a conducive and supportive learning environment that fosters positive student emotions. A pleasant atmosphere and effective teaching strategies can help students stay motivated and engaged.
    \item \textbf{For students}: Students are encouraged to develop resilience and discipline in managing their emotions. Even when experiencing a bad mood, students should try to maintain productive study habits to avoid the negative impact on their academic performance.
\end{itemize} 

\noindent The research ultimately supports the importance of addressing both emotional well-being and learning processes to improve academic outcomes. Although positive mood did not show the expected benefits, ensuring students have strong study habits and the emotional support they need remains essential for their success.

\subsection{Concluding Insights on Student Well-Being in Higher Education}

The exploration of various university research studies on student well-being reveals a complex landscape of mental health challenges faced by students in higher education. Douwes et al. \cite{research-1} and Liu et al. \cite{research-2} emphasize the importance of understanding student well-being from their own perspective. Their research highlights how academic expectations, workload pressures, and examination stressors cause significant psychological fluctuations, demonstrating the need for effective strategies to help students monitor and reflect on their emotional states.\vspace{5mm} \\
Similarly, Febrilia et al. \cite{research-3} explore the link between emotional well-being and academic outcomes, showing that negative moods can severely impact focus and engagement. These insights indicate the critical role of understanding emotional barriers to learning and the potential benefits of tools that track such changes.\vspace{5mm} \\
Furthermore, the “student well-being model” proposed by Soutter et al. \cite{research-4} and the findings by Hascher \cite{research-5} validate the need for comprehensive well-being indicators and assessment tools. Hascher’s study particularly suggests the value of multi-faceted questionnaires and emotion diaries, which can help students better assess their emotional states and academic pressures.\vspace{5mm} \\
Overall, these studies emphasize the necessity of a powerful framework for understanding and addressing the mental health issues of students, which could lead to better support strategies and improved academic outcomes. With this understanding in place, the next step involves applying these insights to design a practical solution for monitoring and supporting student well-being.

\subsection{From the Problem to the Solution}

Building upon these research findings, it is clear that the need for an effective solution to support student well-being is mandatory. In response to this, I decided to design and implement an application, called \textbf{Mood tracker}, aimed at solving the problem of poor mood and well-being among university students. This application will track and project students' emotional states in real time, providing them with tools for self-reflection and helping them be aware of their mood patterns. By enabling students to monitor their course of emotions, the app will empower them to take significant steps toward improving their mental health.

\section{Preparatory Survey}

Despite the relevant research that reviewed existing studies on student well-being and explored the key factors contributing to mental health challenges in universities, it became clear that a custom survey was needed to gain deeper insights into our specific context. While previous researches provided valuable information about the general state of well-being in higher education, it lacked a targeted focus on the influence of teachers, courses, and the university environment on student mental health. To address this gap, I designed a survey to gather direct feedback from students about these aspects.

\subsection{Procedure}

The participants involved in the study were students from various universities and age groups, selected through a convenience sampling method, which included friends and relatives, to represent a versatile view of the university environment. A questionnaire survey was designed, comprising a total of 17 questions (12 mandatory and 5 optional), focusing on different aspects of the student experience. The topics covered by the survey included:

\begin{itemize}
    \item \textbf{Personal Information:} Basic details like the participant’s field of study and current academic year.
    \item \textbf{Academic Experience:} Ratings of overall satisfaction with the academic experience, workload, and feedback on assignments, assessments, or exams.
    \item \textbf{Extracurricular Activities:} Participation in extracurricular activities and their impact on the overall university experience.
    \item \textbf{Mental Health Resources:} Awareness and utilization of mental health resources available on campus.
    \item \textbf{Communication Dynamics:} The preferred communication channels with professors, frequency of communication, and obstacles encountered when interacting with faculty members.
    \item \textbf{Improvements and Suggestions:} Ideas for enhancing the academic experience, the feedback process, and any additional features or tools that could be implemented to engage students better.
\end{itemize}

\noindent The questionnaire was distributed to participants through social media platforms, such as Instagram, Messenger, and WhatsApp, with a completion deadline of one week. Participants were informed that the survey was anonymous and that their responses would be used solely for research purposes.

\subsection{Results}
Eleven students responded to the questionnaire, and in the following section, we discuss the key findings for each topic covered in the survey.

\vspace{5mm}

\noindent \textbf{Field and Year of Study of the participants} \\
The majority of participants are enrolled in Electrical and Computer Engineering indicating a strong representation in this field. Business Engineering and Architecture each constitute of the respondents, showcasing a diverse academic landscape within the surveyed population. Also the fifth academic year is predominant, followed by the third and the sixth years. This distribution may suggest a concentration of survey participants in the latter stages of their academic journey.

\vspace{5mm}

\noindent \textbf{Academic Experience} \\
A considerable number of respondents ($\approx$ 5/10 people) reported a satisfaction rating of 7/10 for their overall academic experience, while approximately 3/10 people reported a rating of 5/10, which means that they are neither satisfied nor dissatisfied. The bulk of participants also considered the workload to be quite substantial, with roughly 5/10 individuals rating it as an 8/10, highlighting the considerable challenges students face in managing their academic responsibilities. We can also observe that most people, around 8/10, take part in after-school activities, and they believe these activities are beneficial. This indicates that being involved in these extra activities not only helps with their studies but also supports overall personal growth. Additionally, 5 out of 10 people expressed dissatisfaction with the feedback mechanisms for communication with professors, indicating potential areas for improvement in this aspect.\vspace{5mm} \\
Notably, a significant proportion of respondents ($\approx$ 9/10 people) identified limited office hours and cited a lack of response to emails as obstacles when trying to communicate with professors. Furthermore, the survey brought to light that  of students are not aware of or have not utilized mental health resources on campus, emphasizing the importance of enhancing awareness and accessibility to support services for the well-being of students. Furthermore, near half of the participants ($\approx$ 4/10 people) reported low numbers about the feedback they receive from their professors about assignments, assessments, and exams, underscoring the need for more comprehensive feedback mechanisms to support students' academic growth and development. According to the contestands, academic stress stems from diverse sources, including the pressure of exams, particularly retakes, heavy workloads, simultaneous deadlines for exams and projects, uncertainties about the future and chosen fields, and concerns about study methods.

\vspace{5mm}

\noindent \textbf{Communication Dynamics} \\
The survey data highlights a significant reliance on feedback from surveys among students, with approximately 5/10 indicating a preference for this communication method over email, online platforms, and in-person office hours. However, despite this preference, half of the students express dissatisfaction with the current feedback mechanisms for communication with professors. This apparent contradiction suggests that while surveys may be a commonly used tool, there might be room for improvement in the effectiveness and responsiveness of the feedback channels.\\ \\
Furthermore, the data indicates a concerning trend regarding student-professor interaction. A majority of students ($\approx$ 8/10) do not feel the need to communicate with their professors, attributing it to a lack of connection. This lack of connection may be influenced by factors such as limited office hours (voted by 10 out of 11 students) and a perceived lack of response to emails ($\approx$ 8/10). The overwhelming use of email as the primary mode of communication ($\approx$ 9/10) suggests that despite its prevalence, students may be facing challenges in establishing meaningful connections and receiving timely support from their professors. Addressing these issues could potentially enhance student engagement and overall satisfaction with the communication channels.

\vspace{5mm}

\noindent \textbf{Improvements and Suggestions} \\
The students provided insightful suggestions for improving their academic experience. A common theme is the need for better support and the normalization of struggles. Students expressed a desire for more advertised support systems and a less stressful environment, suggesting that school hours should be treated as work hours, allowing for a balance between academic and personal time. Another prevalent suggestion is for more transparency in scheduling and a shift towards hands-on learning, with exams conducted through projects rather than traditional methods. Students also emphasized the importance of better organization, materials, and professor approachability, with a call for a rethinking of the teaching process, including more hands-on experiences and the incorporation of young educators. In terms of feedback, students advocated for more active participation and consideration of survey results, hoping for a systematic and regular feedback process that leads to tangible improvements. Lastly, they suggested the implementation of features such as more frequent feedback, increased use of existing tools like Zoom and Skype, and availability of online office hours through apps for better student engagement. These suggestions collectively highlight the students' desire for a more supportive, practical, and engaging academic environment.\vspace{5mm} \\
One of the participants, a student from Germany, said that:

\vspace{5mm}

\noindent Translated Text from German to English \\
\textit{``At my university there are at least a few offers that support students with mental health problems and offer help with certain problems. Some of these offers are provided by the university itself and some are organized as university groups. Although this primarily has to do with the students independently of the teachers, it certainly offers a good opportunity to strengthen the well-being of students. Above all, the advertising and awareness of the offers would have to be increased in order to make the topic of mental well-being much less taboo.''}

\vspace{5mm}

\noindent \textbf{Additional Thoughts} \\
The students express a variety of perspectives on their satisfaction with university settings. One student appreciates the mental health support services available at their university, emphasizing the importance of raising awareness to destigmatize discussions around well-being. Another underscores the need to foster a sense of community among students to enhance their overall happiness in the university environment. On the other hand, some students highlight practical concerns, such as the desire for more office hours for both teachers and secretaries, improved infrastructure, and larger amphitheatres. Additionally, there is a shared sentiment that the university authorities may not fully grasp the challenges faced by the student body, whether in terms of their numbers or the need for responsive actions. These diverse perspectives shed light on various dimensions of student satisfaction, from mental health support to community building and infrastructure improvements, suggesting that a holistic approach is necessary for an optimal university experience.

\subsection{Conclusion}
The results of the survey provide a clearer understanding of the significance of well-being in university environments. The feedback from students reveals a common concern that universities are not addressing mental health issues sufficiently, which are often caused or exacerbated by academic pressures. Many students highlighted that professors and course structures contribute to increased stress, anxiety, and even depression, yet there appears to be a lack of effective measures to counter these issues. For example, students frequently cited heavy workloads, deficient feedback, and limited availability of professors as factors impacting their mental health. Additionally, limited awareness and access to mental health resources on campus were noted as significant barriers to receiving support. This insight emphasizes the need for an effective tool that can assist students in managing their mental health, as current university support systems are perceived as insufficient.\vspace{5mm} \\
Therefore, it is crucial to design and implement a tool that can help students navigate their daily struggles by tracking and understanding their moods. Such a tool would not only provide support but also serve as a guide to enhance their overall university experience. So I decided to design and implement an application aimed at solving the problem of poor mood and well-being among university students. This application can track and project students' emotional states in real time, providing them with tools for self-reflection and helping them be aware of their mood patterns. By enabling students to monitor their course of emotions, the app can empower them to take significant steps toward improving their mental health.

\section{Comparative Analysis of Existing Mood-Tracking Apps}

Given our decision to create an application, it is essential to conduct research on existing state-of-the-art mood-tracking applications that are both popular and well-designed. This research help us gain a deeper understanding of the features that are already available, which we can incorporate into our own application while adding new and innovative elements. To achieve this, I review some of the leading mood-tracking and mental health applications of 2023. These applications showcase a wide range of functionalities and serve as valuable references for identifying current trends in mood tracking.\vspace{5mm} \\
In addition to my own analysis, I include reviews from specialized mental health professionals and app testers. These expert opinions can provide an overall view of each application’s strengths and areas for improvement, offering diverse perspectives on user experience, functionality, and overall design. A brief overview of these applications is summarized in the table below, followed by a detailed comparative analysis based on user feedback, expert reviews, and my personal evaluation.

\FloatBarrier
\begin{table}[ht]
\centering
\begin{tabular}{|p{2cm}|p{6cm}|p{6cm}|}
\hline
\textbf{App} & \textbf{Purpose} & \textbf{Main Features} \\ \hline
Moodfit & A mental health app aimed at improving mood and managing symptoms of stress, anxiety, and depression. Supports users in building healthy habits through CBT-based tools. & Mood journal, daily goals, breathing exercises, mindfulness meditation, medication tracking, lifestyle tracking, therapy companion, summary reports. \\ \hline
Worry Watch & A CBT-based app designed to help users manage anxiety by journaling worries and using mindfulness-based coping tools. & Guided journaling, coping tools, thought diary, affirmations, guided meditation, reminders, privacy controls. \\ \hline
MoodTools & A self-guided mHealth app aimed at managing depression symptoms, with tools for self-monitoring and cognitive restructuring. & PHQ-9 depression test, thought diary, activities tool, videos, safety plan, psychoeducation. \\ \hline
eMoods & A specialized bipolar mood tracker that helps users log mood states, track symptoms, and manage bipolar disorder. & Mood tracking, behavioral logging, symptom rating, graphical reports, custom notes, report generation, daily reminders. \\ \hline
MoodKit & A CBT-based app designed to help users manage depression, anxiety, and stress through structured activities and journaling. & Activities, thought checker, mood tracking, journal, thrive tips, goal tracking. \\ \hline
Daylio & A mood-tracking and journaling app that uses icons and lists to help users track moods and habits quickly without extensive writing. & Mood tracking, activity and habit tracking, statistics hub, customization options, reminders, data exporting, achievements and streaks. \\ \hline
\end{tabular}
\caption{Overview of Mental Health and Mood-Tracking Apps}
\label{tab:overview_mood_apps}
\end{table}
\FloatBarrier

\subsection{Moodfit}

\textbf{\href{https://www.getmoodfit.com/}{Moodfit}\footnote{Link: \url{https://www.getmoodfit.com/}}} is a mental health and wellness app designed to help users improve their mood and manage symptoms of mental health issues like stress, anxiety, and depression \cite{moodfit-review}. Recognized as the ``Best Overall Mental Health App of 2020'', Moodfit offers a variety of tools aimed at promoting emotional well-being. These tools include cognitive behavioral therapy (CBT)-based journaling, daily goal tracking, mindfulness meditation, breathing exercises, and medication tracking. The app supports users in building healthy habits and provides insights into lifestyle factors like sleep, exercise, and nutrition, all of which contribute to mental health.\vspace{5mm} \\
Users can track their emotions with the mood journal feature, allowing them to monitor their emotional patterns and share insights with therapists for a better understanding of their mental state. The app also offers personalized reports, progress charts, and reminders to help users stay on track with their goals. While the app is free to download, premium features are available through in-app purchases.\vspace{5mm} \\
Overall, Moodfit is praised for its well-designed interface, personalized mental health tools, and supportive features like notifications and inspirational articles. However, it does have a few drawbacks, such as the lack of some advanced features and occasional slow performance. Despite these limitations, Moodfit remains a valuable and cost-effective tool for anyone seeking to improve their mental health through daily self-care practices.\vspace{5mm} \\
Here is a summary of the features offered by the application:\vspace{5mm}

\FloatBarrier
\begin{table}[ht]
\centering
\begin{tabular}{|p{4cm}|p{10cm}|}
\hline
\textbf{Feature} & \textbf{Description} \\ \hline
Mood Journal & Allows users to track their mood daily and process their thoughts using a journal based on cognitive behavioral therapy (CBT) techniques. \\ \hline
Daily Goals and Self-care & Encourages users to set daily mental health goals and practice self-care activities that promote well-being. \\ \hline
Breathing Exercises & Provides guided breathing exercises to reduce stress and anxiety, improving emotional regulation. \\ \hline
Gratitude Journal & Encourages users to log things they are grateful for, helping to foster a more positive mindset. \\ \hline
Mindfulness Meditation & Offers mindfulness-based meditation exercises that help users stay present and calm. \\ \hline
Medication Tracking & Helps users track their medications and the effects or side effects related to mood or mental health. \\ \hline
Sleep and Lifestyle Tracking & Allows users to monitor aspects of their lifestyle like sleep, nutrition, and exercise, and how these factors affect their mood. \\ \hline
Therapy Companion & Enables users to share mood charts and progress reports with therapists or coaches to assist in therapy sessions. \\ \hline
Summary Reports & Provides users with downloadable weekly or monthly reports to track their progress and identify patterns in their behavior and mood. \\ \hline
\end{tabular}
\caption{Summary of Features in the Moodfit App}
\label{tab:moodfit_features}
\end{table}
\FloatBarrier

\subsection{WorryWatch}

The \textbf{\href{https://worrywatch.com/}{Worry Watch}\footnote{Link: \url{https://worrywatch.com/}}} app is designed as a cognitive behavioral therapy (CBT)-based tool to help users manage anxiety by journaling their worries, identifying cognitive distortions, and using mindfulness-based coping skills \cite{worrywatch-review}. The review, written by a licensed psychologist, provides insight into the app's effectiveness over a month of testing. The app's features include guided journaling prompts, anxiety tracking, coping mechanisms such as breathing exercises, grounding techniques, and guided meditations.\vspace{5mm} \\
The reviewer appreciated the app's simple and customizable features, such as the breathing exercises and mindfulness tools, which helped in managing daily anxiety. However, concerns were raised regarding the lack of involvement of mental health professionals in the app's development. The app offers both a free version and a paid version, with the free version having limited features, such as restricting the number of journal entries per day.\vspace{5mm} \\
Despite the lack of professional development input, the reviewer found Worry Watch useful for monitoring thoughts and reducing anxiety symptoms, recommending it as a supplemental tool but advising users to consult with their healthcare providers for better guidance.\vspace{5mm} \\
Here is a summary of the features offered by the application:\vspace{5mm}

\FloatBarrier
\begin{table}[ht]
\centering
\begin{tabular}{|p{4cm}|p{10cm}|}
\hline
\textbf{Feature} & \textbf{Description} \\ \hline
Guided Journaling & Allows users to document their worries and thoughts, identify anxiety triggers, and reflect on cognitive distortions such as ``all or nothing'' thinking and ``mind reading''. \\ \hline
Coping Tools & Includes customizable breathing exercises and grounding techniques that guide users through using their five senses to reduce anxiety. \\ \hline
Thought Diary & Helps users log their anxious thoughts and reflect on their effectiveness in handling the worry, encouraging them to assess outcomes and cognitive distortions. \\ \hline
Affirmations & Provides users with positive affirmations and inspirational quotes, with the ability to create custom affirmations to reinforce healthy thought patterns. \\ \hline
Guided Meditation & Offers guided imagery meditation to promote mindfulness and relaxation, with options to customize session durations between 10 to 60 minutes. \\ \hline
Reminders and Notifications & Sends users reminders to check in, journal, or practice coping skills, helping them stay consistent with mental health activities. \\ \hline
Privacy Controls & Ensures data security by allowing users to delete their entries and confirming that no personal data is tracked or shared by the developers. \\ \hline
\end{tabular}
\caption{Summary of Features in the Worry Watch App}
\label{tab:worrywatch_features}
\end{table}
\FloatBarrier

\subsection{MoodTools}

The study examines the user engagement and behavior associated with \textbf{\href{https://www.moodtools.org/}{MoodTools}\footnote{Link: \url{https://www.moodtools.org/}}}, a self-guided mobile health (mHealth) app designed to assist users with managing symptoms of depression \cite{moodtools-review}. MoodTools includes tools like the Patient Health Questionnaire (PHQ-9), a Thought Diary for cognitive restructuring, and resources such as behavioral activation techniques, videos, and a safety plan feature for crisis situations. The study analyzes data from 158,930 users across 198 countries, demonstrating the global interest in MoodTools, even in countries where English is not the primary language.\vspace{5mm} \\
Results from the study show that most users engage with the app for brief periods, with typical users completing three sessions over 90 days, spending a total of about 12 minutes. The most frequently visited tools were the Thought Diary and PHQ-9 test, indicating users gravitated towards self-monitoring and cognitive restructuring. However, the study noted the challenge of retaining users, as more than half did not return after the first session. This indicates that while the app attracts a global audience, maintaining consistent user engagement is difficult without external motivation or guidance.\vspace{5mm} \\
The research highlights the potential of self-guided mHealth apps like MoodTools to help alleviate the global burden of depression, especially in regions with limited access to mental health resources. However, it also emphasizes the need for further research into how these apps can effectively retain users and improve long-term outcomes in managing depression symptoms.\vspace{5mm} \\
Here is a summary of the features offered by the application:\vspace{5mm}

\FloatBarrier
\begin{table}[ht]
\centering
\begin{tabular}{|p{4cm}|p{10cm}|}
\hline
\textbf{Feature} & \textbf{Description} \\ \hline
PHQ-9 Depression Test & A 9-item questionnaire that helps users assess the severity of their depression symptoms and monitor changes over time. \\ \hline
Thought Diary & A cognitive restructuring tool that allows users to record negative thoughts and reframe them to promote positive thinking, based on cognitive behavioral therapy (CBT) techniques. \\ \hline
Activities Tool & Encourages users to engage in behavioral activation by suggesting activities that can improve mood, while allowing users to track which activities are most beneficial. \\ \hline
Videos & A curated collection of YouTube videos, including guided meditations, TED Talks, and relaxing sounds for mindfulness and emotional well-being. \\ \hline
Safety Plan & Provides users with a tool to create a personal safety plan for managing suicidal thoughts, with quick access to local emergency services and national crisis hotlines. \\ \hline
Psychoeducation & Offers educational resources about depression, helping users better understand their condition and ways to manage symptoms. \\ \hline
\end{tabular}
\caption{Summary of Features in the MoodTools App}
\label{tab:moodtools_features}
\end{table}
\FloatBarrier

\subsection{eMoods (Bipolar Mood Tracker)}

The \textbf{\href{https://emoodtracker.com/}{eMoods (Bipolar Mood Tracker)}\footnote{Link: \url{https://emoodtracker.com/}}} app is a specialized tool designed to help individuals manage and monitor symptoms of bipolar disorder \cite{emoods-review}. This app allows users to track key mood states, including depressed and elevated moods, anxiety, and irritability, on a 4-point scale ranging from none to severe. Additionally, users can log important behavioral data, such as hours of sleep, medication intake, and whether verbal therapy was received. A key feature of the app is the ability to generate daily mood entries, providing a comprehensive view of mood fluctuations over time. These entries are visually represented in a color-coded graph, helping users and their healthcare providers identify patterns and correlations between mood states and other behaviors, such as medication adherence. The app also supports the creation of summary reports in PDF or CSV format, which can be shared with therapists for more effective care management. However, the app's customization options are somewhat limited, as it does not allow users to add symptoms or behaviors beyond the preset categories, though a notes section provides space for additional comments. Overall, eMoods is a valuable tool for individuals managing bipolar disorder, offering an easy way to track and report mood changes while supporting the treatment process.\vspace{5mm} \\
Here is a summary of the features offered by the application:\vspace{5mm}

\FloatBarrier
\begin{table}[ht]
\centering
\begin{tabular}{|p{4cm}|p{10cm}|}
\hline
\textbf{Feature} & \textbf{Description} \\ \hline
Mood Tracking & Allows users to track mood states, including depressed, elevated, anxiety, and irritability, rated on a 4-point scale from none to severe. \\ \hline
Behavioral Logging & Enables tracking of sleep hours, therapy sessions, and medication intake to identify how these factors relate to mood changes. \\ \hline
Symptom Rating & Users can log the presence or absence of psychotic symptoms, offering a fuller picture of mental health on a daily basis. \\ \hline
Graphical Reports & Daily mood entries are represented in a color-coded graph, showing mood fluctuations alongside medication and other behavioral factors. \\ \hline
Custom Notes & Includes a notes section where users can log additional symptoms or events, allowing for some customization beyond the preset options. \\ \hline
Report Generation & Users can generate detailed reports in PDF or CSV format to share with healthcare providers, enhancing treatment collaboration. \\ \hline
Daily Reminders & Provides reminders to log moods and other behaviors at a scheduled time, helping users maintain consistency in tracking. \\ \hline
\end{tabular}
\caption{Features of the eMoods Bipolar Mood Tracker App}
\label{tab:emoods_features}
\end{table}
\FloatBarrier

\subsection{MoodKit}

\textbf{\href{https://crediblemind.com/apps/moodkit}{MoodKit}\footnote{Link: \url{https://crediblemind.com/apps/moodkit}}} is a Cognitive Behavioral Therapy (CBT)-based mental health app designed to help users manage symptoms of depression, anxiety, and stress \cite{moodkit-review}. It offers a comprehensive set of tools to enhance emotional well-being through a structured approach. The app is divided into four main sections: Activities, Thoughts, Mood, and Journal. The Activities section encourages users to engage in positive habits across categories like productivity, social interaction, and physical health, with a focus on setting goals and tracking progress. The Thoughts section is based on CBT principles, allowing users to identify and challenge cognitive distortions, such as negative thought patterns. The Mood section enables users to rate their mood multiple times a day, helping track emotional changes over time. In the Journal section, users can reflect on their thoughts and experiences by writing daily entries or adding notes linked to specific activities or mood ratings. Additionally, the app includes ``Thrive Tips'', which provide users with helpful advice for personal growth and stress management. While MoodKit is praised for being rooted in scientific evidence and its effective use of CBT techniques, it requires users to be proactive and self-motivated to fully benefit from its features. The app's text-based nature may also be challenging for users looking for more interactive, game-based experiences.\vspace{5mm} \\
Here is a summary of the features offered by the application:\vspace{5mm}

\FloatBarrier
\begin{table}[ht]
\centering
\begin{tabular}{|p{4cm}|p{10cm}|}
\hline
\textbf{Feature} & \textbf{Description} \\ \hline
Activities & Users can choose from a variety of activities related to productivity, social interaction, physical health, and enjoyment. The app helps set goals and track progress. \\ \hline
Thought Checker & This feature allows users to identify cognitive distortions and challenge negative thinking patterns by analyzing their reactions to stressful situations. \\ \hline
Mood Tracking & Users can log their mood throughout the day, providing a visual representation of mood changes over time. \\ \hline
Journal & The journal feature enables users to write reflective entries or attach notes to specific activities or mood events. \\ \hline
Thrive Tips & Provides practical advice and tips for personal growth, emotional well-being, and stress management. \\ \hline
Goal Tracking & Encourages users to set and track goals related to mental health, including achieving specific activities and improving emotional states. \\ \hline
\end{tabular}
\caption{Summary of Features in the MoodKit App}
\label{tab:moodkit_features}
\end{table}
\FloatBarrier

\subsection{Daylio}

\textbf{\href{https://daylio.net/}{Daylio}\footnote{Link: \url{https://daylio.net/}}} is a user-friendly journaling and mood-tracking app designed for individuals who want to track their emotions, habits, and activities in a quick, visual format \cite{daylio-review}. Rather than relying on traditional text-based journaling, Daylio allows users to log their moods and activities through icons and bulleted lists, making it a great fit for those who may not enjoy writing. The app offers customizable mood icons and activity categories, allowing users to tailor their experience to their personal preferences. Over time, Daylio generates detailed statistics, providing insights into how specific activities and habits influence mood. The app includes both free and premium versions, with the premium offering advanced features such as PDF exports, more mood icons, and additional statistics. Daylio's straightforward design and easy tracking make it ideal for users with busy schedules or those looking for a non-writing-based journaling method. However, its lack of journal prompts and community features may not appeal to users seeking a more in-depth or interactive experience.\vspace{5mm} \\
Here is a summary of the features offered by the application:\vspace{5mm}

\FloatBarrier
\begin{table}[ht]
\centering
\begin{tabular}{|p{4cm}|p{10cm}|}
\hline
\textbf{Feature} & \textbf{Description} \\ \hline
Mood Tracking & Allows users to log their mood using customizable icons, with options to track multiple moods throughout the day. \\ \hline
Activity and Habit Tracking & Users can select from a list of activities and habits, such as sleep, hobbies, and social events, and track how these influence their mood. \\ \hline
Statistics Hub & Provides visual reports and mood charts, helping users identify trends and patterns in their emotional states and activities. \\ \hline
Customization Options & Offers users the ability to customize mood icons, activity categories, and color schemes to personalize their experience. \\ \hline
Reminders and Goals & Users can set daily reminders to log moods and track progress towards personalized goals, such as improving sleep or physical activity. \\ \hline
Data Exporting & In the premium version, users can export their data as a PDF or CSV file, useful for sharing with a therapist or keeping detailed records. \\ \hline
Achievements and Streaks & Tracks the user's progress by showing streaks of consecutive days logged and offering achievements for consistency. \\ \hline
\end{tabular}
\caption{Summary of Features in the Daylio App}
\label{tab:daylio_features}
\end{table}
\FloatBarrier

\subsection{Conclusion}

In reviewing the most popular mood-tracking apps of 2023, several valuable insights emerged that guide the development of my own application, each app offering distinct strengths and areas for improvement. Moodfit, despite having a somewhat inconsistent interface, provides excellent tools like charts and meditation exercises that could greatly enhance the user experience. However, its slow performance and the absence of some key features, as well as the need for more privacy settings, underscore areas where the application can stand out. Worry Watch, though limited to iOS and hindered by a challenging interface, offers valuable affirmations and customizable breathing and meditation exercises, fostering a safe and supportive environment for users — something I aim to replicate. MoodTools, with its simple design, highlights the need for a more polished and interactive frontend, a feature that the app can emphasize to boost engagement and user satisfaction. eMoods (Bipolar Mood Tracker), being a more specialized tool for bipolar disorder, features a subscription-based model and a somewhat complex interface. This points to the importance of creating a user-friendly and affordable app that appeals to a broader audience. MoodKit, despite an outdated interface, offers password-protected journals and comprehensive mood-tracking charts, reinforcing the importance of prioritizing security and robust mood monitoring. Finally, Daylio stands out for its simplicity and engaging color scheme, along with its innovative use of emojis for mood tracking, which aligns with the visual and interactive approach I plan to incorporate.\vspace{5mm} \\
In summary, the unique features and thoughtful designs of these apps, such as the emphasis on mental well-being, security, and ease of use, directly influence the development of my own application. By integrating the best elements of these standout products and addressing their shortcomings, this app promises to deliver an enhanced, user-friendly, and engaging experience.

\section{Summary}

In this chapter, we explored the various mental health challenges students encounter daily. Following an extensive review of existing literature, I conducted my own survey to gather insights directly from my peers on the well-being issues they face in university life. Recognizing the need for additional support, I concluded that an application should be developed to assist students in their academic journey. To gain a clearer picture of the current condition, I also reviewed the latest mood-tracking tools available in 2023. Through this research, it became clear how crucial it is to prioritize and address students' mental health and well-being.\vspace{5mm} \\
By understanding the significance of student well-being, I aim to create an application that helps students overcome these challenges and fosters a more positive university experience. In the next chapter, we outline the methodology for developing an application that meets these needs. This includes discussing the application's intended use and defining its core features based on identified requirements. This structured approach guides the design and development process, ensuring that the final prototype effectively supports students in tracking their mood and managing their overall well-being.