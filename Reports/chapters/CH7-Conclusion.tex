\chapter{Conclusion}

This thesis presents a comprehensive study on enhancing student mental health in higher education through the development of a mood-tracking application. The research focused on identifying key factors influencing well-being and implementing a system that could monitor these variables through daily surveys and assessments. The application was designed with user-centered principles, including intuitive interfaces and seamless data management. The proposed solution is adaptable to various university environments, providing valuable insights into students' emotional states and contributing to a more supportive academic community.

\section{Synopsis of the thesis}

The thesis followed a structured approach to address the problem of student mental health by developing an application that provides mood tracking and support. Initially, a detailed literature review was conducted to identify the core issues and understand existing solutions in the field. This was followed by an in-depth analysis using frameworks like the PACT model, along with personas and user scenarios, which laid the groundwork for understanding user needs and defining the application’s initial direction.\vspace{5mm} \\
The design phase focused on defining the user experience, incorporating elements like colors, fonts, and interface layouts to create a user-friendly environment. Design inspirations were drawn from existing solutions, and mockups and prototypes were developed for each screen of the application. The aim was to ensure a complete and intuitive experience that aligns with user expectations.\vspace{5mm} \\
During the implementation phase, the database was designed using ERD and UML diagrams to structure the data efficiently, the Xata platform for implmenting the application's database, while the front-end and back-end were developed using appropriate frameworks and technologies, such as React Native and Express respectively. The deployment was handled through cloud services, ensuring scalability and stability of the system.\vspace{5mm} \\
Finally, the application was evaluated using both expert-based and user-based methods. A specialized questionnaire (UEQ) and structured interviews were conducted to gather feedback, which was then used to assess the effectiveness and usability of the application. The evaluation highlighted areas for improvement and validated the system’s potential to support student well-being effectively.

\section{Future Steps}

Based on the evaluation of the application, valuable feedback and suggestions for improvement have been collected. Future developments may include refining the application’s functionality and design, adding some new features which were proposed by the users during their interviews, optimizing deployment for platforms like the App Store and Play Store, and potentially offering the application as a resource for universities to support student mental health. Expanding the application's reach and usability will contribute to better engagement and effectiveness in addressing mental health challenges in higher education environments.