\clearpage
\myemptypage
\thispagestyle{empty}

\renewcommand{\abstractname}{\large Abstract}
\begin{abstract}
    In this diploma thesis, a full-stack application, named \textbf{Mood Tracker}, is designed and developed with the purpose of supporting the mental health of university students. The research and development process of the application is based on the principles of Human-Computer Interaction (HCI) and, more specifically, on creating a human-centered design that enhances user experience and usability. The application aims to provide students with a platform for mood tracking and emotional self-reflection, contributing to better understanding and management of their mental health.\vspace{2mm} \\
    The development of the application followed a systematic approach, starting with a detailed literature review to identify the core issues and existing solutions. This was followed by a design phase where personas, user scenarios, and interface mockups were created to define the application's layout and functionalities. The project then proceeded to implementation, covering database structuring and the development of front-end and back-end components, ensuring a seamless and integrated system.\vspace{2mm} \\
    The user experience of the application was evaluated using methods from the human-centered design framework, such as interviews and questionnaires, which provided valuable feedback throughout its development. Upon completion, the resulting prototype is a complete tool that offers university students an accessible means to understand their emotional well-being, fostering self-awareness and promoting a healthier lifestyle in the academic setting.\vspace{5mm} \\
    \textbf{Keywords}: Students' mental health, Mood tracking, Human-Computer Interaction, Human-centered design, Cross-platform application, React Native
\end{abstract}




\clearpage
\myemptypage
\thispagestyle{empty}

\renewcommand{\abstractname}{\large Σύνοψη}
\begin{abstract}
    Στην παρούσα διπλωματική εργασία, σχεδιάζεται και αναπτύσσεται μια ολοκληρωμένη εφαρμογή με το όνομα \textbf{Mood Tracker}, με σκοπό την υποστήριξη της ψυχικής υγείας των φοιτητών. Η διαδικασία ανάπτυξης βασίζεται στις αρχές της Αλληλεπίδρασης Ανθρώπου-Υπολογιστή (HCI) και συγκεκριμένα στη δημιουργία ενός ανθρωποκεντρικού σχεδιασμού που βελτιώνει την εμπειρία του χρήστη και τη χρηστικότητα της εφαρμογής. Η εφαρμογή στοχεύει στην παροχή μιας πλατφόρμας για την καταγραφή διάθεσης και τον συναισθηματικό αυτο-στοχασμό, συμβάλλοντας στην καλύτερη κατανόηση και διαχείριση της ψυχικής υγείας των φοιτητών.\vspace{2mm} \\
    Η ανάπτυξη της εφαρμογής ακολούθησε μια συστηματική προσέγγιση, ξεκινώντας από μια αναλυτική βιβλιογραφική ανασκόπηση για τον εντοπισμό των βασικών προβλημάτων και των υπαρχουσών λύσεων. Στη συνέχεια, ακολούθησε η φάση σχεδιασμού, όπου δημιουργήθηκαν περσόνες, σενάρια χρήσης και πρωτότυπα διεπαφών, καθορίζοντας τη διάταξη και τη λειτουργικότητα της εφαρμογής. H εργασία προχώρησε στην υλοποίηση, καλύπτοντας τον σχεδιασμό της βάσης δεδομένων, καθώς και την ανάπτυξη του front-end και του back-end, διασφαλίζοντας ένα ολοκληρωμένο και ενοποιημένο σύστημα.\vspace{2mm} \\
    Η εμπειρία χρήστη αξιολογήθηκε χρησιμοποιώντας μεθόδους του ανθρωποκεντρικού σχεδιασμού, όπως συνεντεύξεις και ερωτηματολόγια, που παρείχαν πολύτιμες πληροφορίες καθ’ όλη τη διάρκεια της ανάπτυξης. Με την ολοκλήρωση της εργασίας, το τελικό πρωτότυπο αποτελεί ένα πλήρες εργαλείο που προσφέρει στους φοιτητές ένα εύχρηστο μέσο για να κατανοούν την ψυχική τους κατάσταση, προάγοντας την αυτογνωσία και έναν πιο υγιή τρόπο ζωής στο πανεπιστημιακό περιβάλλον.\vspace{5mm} \\
    \textbf{Λέξεις-κλειδιά}: Ψυχική υγεία φοιτητών, Παρακολούθηση διάθεσης, Αλληλεπίδραση Ανθρώπου-Υπολογιστή, Ανθρωποκεντρικός σχεδιασμός, Εφαρμογή cross-platform, React Native
\end{abstract}

\clearpage
\myemptypage
\thispagestyle{empty}