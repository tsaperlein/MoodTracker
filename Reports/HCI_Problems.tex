\section{Key Problems}

Possible key problems with my app are presented and explained in detail below. The problems were categorized into problems that should be solved by future design and problems that arise through human-computer interaction.

\subsection{Problems to be Solved by Future Design}

\begin{description}

    \item[Credibility] The trustworthiness of users in the application poses a potential challenge. Specifically, the anonymization of data might instigate concern among users, as they cannot ascertain whether the data has been anonymized accurately. The ambiguity surrounding the possibility of tracing back information in case of negative feedback from a student could lead to inflated survey results, painting a more favorable picture than the actual reality.
    
    \item[Honesty/Seriousness] Another issue may arise in the manner users approach surveys. There is a possibility that some users may not approach the surveys earnestly, providing inaccurate responses or withholding information when facing challenges with assessments, exams, or professors.
    
    \item[Visualization of the Survey Results] A critical consideration is the effective presentation of survey results. Ensuring results are easily comprehensible and transparent is essential, with a careful approach to prevent distortion based on the presentation method. 
    
    \item[Questions and Answers] The nature of the questions also presents a challenge. Various data collection methods, such as multiple-choice, single-choice, sliders, or grading scales to gauge agreement with a statement, offer diverse avenues. Particularly with the latter, it's crucial to recognize that responses might carry inherent biases. For instance, on a grading scale, users may tend towards more favorable ratings. Consequently, selecting answer options in sentiment surveys demands meticulous attention. 
    
    \item[Application Constraints] Beyond the aforementioned concerns, it's imperative to acknowledge the inherent limitations of the application. Personal issues affecting students outside the university realm may be intricate and go beyond the scope of the app's capabilities. Identifying such problems and providing meaningful assistance in finding solutions can be challenging for the application.
    
\end{description}

\subsection{Possible HCI Problems}

\begin{description}
    
    \item[Simplicity of the App] The app's design should embrace simplicity, ensuring that users are not inundated with complexity, enabling easy comprehension and utilization. If the application's structure becomes overly intricate, there is a possibility of user impatience leading to app abandonment or even deletion.

    \item[Persuasion Dilemma] A primary objective in design is to craft the application in a manner that persuades users to place trust in its functionality and engage with it sincerely. The elements discussed under ``Trustworthiness'' and ``Veracity/ Credibility'' earlier encapsulate precisely this objective.

    \item[Information Visualization Challenge] Effectively presenting information poses a common challenge. This is exemplified in the aforementioned points concerning the ``Visualization of the Survey Results'' and ``Questions and Answers''. Striking the right balance is crucial, ensuring that users are not overwhelmed by excessive information while providing a comprehensive overview of the most crucial details.
    
\end{description}

\section{Conclusion}

Building upon the insights gathered from the aforementioned articles on students' moods in universities and my comprehensive survey aimed at uncovering the most pressing issues affecting their well-being, my mood-tracking application takes a pioneering approach. The information gleaned from these sources has been instrumental in shaping a solution that not only combines the best features from existing apps but also digs into the core challenges faced by students in the university environment.
\newline

The combination of these sources has enabled my application to transcend the limitations observed in current solutions. It aims to provide an unparalleled, user-friendly experience that not only visually captivates users but also serves as a powerful tool for understanding and anonymously tracking students' moods efficiently. My solution is not merely an integration of features; it is a thoughtful response to the intricate interplay of factors influencing the emotional landscape of students in higher education.
\newline

Moreover, students generally perceive the academic demands as notably intense and burdensome. This insight underscores the demanding nature of the academic experience, where students grapple with a rigorous curriculum, facing significant hurdles in their pursuit of academic success.